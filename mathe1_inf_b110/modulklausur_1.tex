% "THE BEER-WARE LICENSE" (Revision 42):
%
% <timklge@wh2.tu-dresden.de> wrote this file. As long as you
% retain this notice you can do whatever you want with this stuff.
% If we meet some day, and you think this stuff is worth it,
% you can buy me a beer in return - Tim Kluge

\documentclass[12pt,landscape]{article}
\usepackage{multicol}
\usepackage{calc}
\usepackage{delarray}
\usepackage{amssymb}
\usepackage[landscape]{geometry}
\usepackage[utf8]{inputenc}
\usepackage{color}
\usepackage[compact]{titlesec}

\pagestyle{empty}
\geometry{top=1cm,left=1cm,right=1cm,bottom=1cm}

\makeatletter

\makeatother
\setcounter{secnumdepth}{0}

\begin{document}

\footnotesize
\begin{multicols}{3}

\begin{center}
     \Large{\textbf{Mathe 1 Cheatsheet}} \\
     \small{Für erstes Mathemodul (Ganter, Noack)}
\end{center}

\section{LAG}
\subsection{Vollständige Induktion}
Beweis erfolgt für $n \geq 1, n \in{\mathbb{N}}$ mittels vollständiger Induktion.\\
\textbf{IV} für $n = 1$: $\sum_{k=1}^N(4k-3) = n(2n - 1) \iff (4*1-3) = 1*(2-1) \iff 1 = 1 \surd$ \\
\textbf{IA} für $n \geq 1$: $\sum_{k=1}^N(4k-3) = n(2n-1)$\\
\textbf{IS} für $n = n + 1:$\\
$\sum_{k=1}^{N+1}(4k-3) = (n+1)(2(n+1)-1) \iff \sum_{k=1}^{N}(4k-3) + (4*(n+1)-3) = (n+1)(2n+1)$\\
\textit{nach IA gilt}:\\
$n(2n-1)+(4*(n+1)-3)=(n+1)(2n+1)\iff
2n^2-n+4n+4-3=(n+1)(2n+1) \iff
(2n^2+3n+1 = 2n^2 + n + 2n + 1 \iff 2n^2 + 3n + 1 = 2n^2 + 3n + 1 \surd$
\subsection{Ebenengleichungen}
In $\mathbb{R}_3$ kann es Ebenen der folgenden Formen geben:
\begin{enumerate}
\item $E: \vec{x} = ToDo$
\end{enumerate}
\subsection{Lineare Gleichungssysteme}
Lösen von linearem Gleichungssystem $A\vec{x}=\vec{b}$ mit Gauss durch elementare Zeilenumformungen (Addieren eines Vielfachen von anderen Gleichungen, Umtauschen von Gleichungen und Skalierung). Erw. Koeffizietenmatrix:
\[\begin{array}({ccc|c})
  A_{1,1} & A_{1,2} & A_{1,3} & \vec{b}_1\\
  A_{2,1} & A_{2,2} & A_{2,3} & \vec{b}_2
\end{array}\]
\subsection{Inverse zu Matrizen}
Es existiert ein $A_{-1}$ zu einer Matrix $A$, wenn $A$ quadratisch ist und das homogene LGS $A * \vec{x} = \vec{0}$ nur die triviale Lösung hat. Dann gilt $A_{-1} \times A = E$.
\textit{Pivotspalten} sind Spalten, in denen nur in einer Zeile eine 1 steht. Gibt es nach Vor- und Rückwärtsphase noch Nicht-Pivotspalten, hat das LGS unendlich viele Lösungen. Gibt es eine Zeile $(0,0,0,c)$ mit $c \neq 0$, gibt es keine Lösung.
\subsection{LU-Faktorisierung}
\begin{itemize}
\item Soll A als untere Dreickecksmatrix $L$ und obere Dreieckmatrix $U$ faktorisiert werden, gilt $A = L \times U$
\item $U$ ist die erste durch elementare Zeilenumformungen erreichte Matrix in Zeilenstufenform (ZSF) (\textbf{nicht} reduziert). Vertauschungen und Skalierungen sind zwecks Eindeutigkeit \textbf{nicht} erlaubt!
\item $L$ ist das Produkt der invertierten Elementarmatrizen $L = E_1^{-1} \times E_2^{-1} \times E_n^{-1}$ bis zum Erreichen von $U$
\item Lösung von $L \times U = \vec{b}$, wenn $L$ und $U$ bekannt sind durch $L \times \vec{y} = \vec{b}$. Dann $U \times \vec{x} = \vec{y}$ lösen.
\end{itemize}
\section{Diskrete Strukturen}
\subsection{Begriffsverbände}
\begin{itemize}
\item Formaler Kontext $G, M, I$ in Tabelle: Merkmale oben, Gegenstände links. 
\item Lesen von Begriffsverband: Von oben nach unten
\end{itemize}
\subsection{Kanonische Darstellung und Teiler}
\begin{itemize}
\item Primfaktorzerlegung ist kanonische Darstellung, bswp. 22: $22 = 2^1 * 11^1$. 3 teilt $n \in \mathbb{N}$, wenn Quersumme durch 3 teilbar ist. 11 teilt $n \in \mathbb{N}$, wenn alternierende Quersumme durch 11 teilbar ist. Bspw. 61259: $6 - 1 + 2 - 5 + 9 = 11 \surd$
\item Anzahl Teiler von $n$: Summe der Exponenten der Primfaktoren, jeweils + 1, bspw. 22: $teilerzahl(22) = (1+1)*(1+1) = 4$ (nämlich 2, 11, 1, 22)
\item Anzahl teilerfremder Zahlen zu $n$: Eulersche $\varphi$-Funktion. Für $n \in \mathbb{N}$ mit den Primfaktoren $p_1^a ... p_k^l$: $\varphi(n)=n*(1-\frac{1}{p_1})*(1-\frac{2}{p_2})*...*(1-\frac{k}{p_k})$ (p sind also immer die Basen)
\item Primzahlen 1-100: 2, 3, 5, 7, 11, 13, 17, 19, 23, 29, 31, 37, 41, 43, 47, 53, 59, 61, 67, 71, 73, 79, 83, 89, 97
\end{itemize}
\subsection{ggT und euklidischer Algorithmus}
ggT wird mit euklidischem Algorithmus bestimmt. In erweiterter Form:
\vspace*{0.5cm}\newline
\begin{tabular}{|c|c|c|c|c|c|c|}
\hline \rule[-2ex]{0pt}{5.5ex} $i$ & $n_i$ & $n_{i+1}$ &  $r$ & $q_i$ & $a_{i+1}$ & $a{_i}$ \\ 
\hline \rule[-2ex]{0pt}{5.5ex} 1. & \textcolor{red}{238} & \textcolor{blue}{154} & 84 & 1 & \textcolor{blue}{2} & \textcolor{red}{-3} \\ 
\hline \rule[-2ex]{0pt}{5.5ex} 2. & 154 &  84 & 70 & 1 & -1 & 2 \\ 
\hline \rule[-2ex]{0pt}{5.5ex} 3. & 84 & 70 & 14 & 1 & 1 & -1 \\ 
\hline \rule[-2ex]{0pt}{5.5ex} 4. & 70 & 14 & 0 & 5 & 0 & 1 \\ 
\hline \rule[-2ex]{0pt}{5.5ex} 5. & 14 & 0 & ggT: 14 &  & 1 & 0 \\ 
\hline 
\end{tabular}
\vspace{0.5cm}\newline
$a_i = a_{i+2} - q_i * a_{i+1}$\\
$q_i = n_i div n_{i+1}$
\subsection{Restklassenringe}
\begin{itemize}
\item In $\mathbb{Z}_{22}$ sind 0-21 drin, negative Zahlen: Solange $22 \equiv 0$ (mod n) addieren, bis Ergebnis in $\mathbb{Z}_{n}$ liegt
\item Einheit: Zahl $a \in \mathbb{Z}_{n}$ ist Einheit, wenn $a * b \equiv 1$ (mod n). Das gilt dann, wenn $ggT(a, n) = 1$.
\item Nullteiler: Zahl $a \in \mathbb{Z}_{n}$ ist Nullteiler, wenn $a * b \equiv 0$ (mod n)
\item Division nur für Einheiten. Ist $a$ Einheit in $\mathbb{Z}_{n}$, dann wird inverses Element $n^{-1}$ durch erw. euklidischen Algorithmus bestimmt (mit $n$ als erstem Wert und $a$ als zweitem). Inverses zu n ist dann der Wert, der als $a_1$ oben rechts rauskommt.
\item Rechnen mit Potenzen: $a^m * a^n = a^{m+n}$, $a^n=a^{n \bmod (p-1)}$ 
\item Lemma von Euler-Fermat: $a^{\varphi(n)} \bmod n = 1$, falls $a$ zu $n$ teilerfremd ist ($ggT(a, n)=1$)
\end{itemize}

\rule{0.3\linewidth}{0.25pt}
\scriptsize

Gebaut mit \LaTeX

\end{multicols}
\end{document}